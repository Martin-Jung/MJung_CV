%%%%%%%%%%%%%%%%%%%%%%%%%%%% Document Setup %%%%%%%%%%%%%%%%%%%%%%%%%%%%

% Don't like 10pt? Try 11pt or 12pt
\documentclass[10pt]{article}
\RequirePackage[T1]{fontenc}

% The automated optical recognition software used to digitize resume
% information works best with fonts that do not have serifs. This
% command uses a sans serif font throughout. Uncomment both lines (or at
% least the second) to restore a Roman font (i.e., a font with serifs).
\usepackage{times}
\renewcommand{\familydefault}{\sfdefault}

% The OCR software also has a hard time with italics. These commands get
% rid of the two common ways to italicize text in LaTeX. Get rid of them
% to turn italics back on.
\renewcommand\emph[1]{#1}
\renewcommand\textit[1]{\underline{#1}}

% This is a helpful package that puts math inside length specifications
\usepackage{calc}

% This package helps LaTeX auto-hyphenate hyphenated words if you use
% special hyphens. For example, bio\-/mimicry will properly hyphenate
% ``mimicry'' if necessary.
\usepackage[shortcuts]{extdash}

% Euro Symbol
\usepackage{eurosym}

% The font awesome icons
\usepackage{fontawesome}

% Layout: Puts the section titles on left side of page
\reversemarginpar

%
%         PAPER SIZE, PAGE NUMBER, AND DOCUMENT LAYOUT NOTES:
%
% The next \usepackage line changes the layout for CV style section
% headings as marginal notes. It also sets up the paper size as either
% letter or A4. By default, letter was used. If A4 paper is desired,
% comment out the letterpaper lines and uncomment the a4paper lines.
%
% As you can see, the margin widths and section title widths can be
% easily adjusted.
%
% ALSO: Notice that the includefoot option can be commented OUT in order
% to put the PAGE NUMBER *IN* the bottom margin. This will make the
% effective text area larger.
%
% IF YOU WISH TO REMOVE THE ``of LASTPAGE'' next to each page number,
% see the note about the +LP and -LP lines below. Comment out the +LP
% and uncomment the -LP.
%
% IF YOU WISH TO REMOVE PAGE NUMBERS, be sure that the includefoot line
% is uncommented and ALSO uncomment the \pagestyle{empty} a few lines
% below.
%

%% Use these lines for letter-sized paper
\usepackage[paper=letterpaper,
            %includefoot, % Uncomment to put page number above margin
            marginparwidth=1.2in,     % Length of section titles
            marginparsep=.05in,       % Space between titles and text
            margin=1in,               % 1 inch margins
            includemp]{geometry}

%% Use these lines for A4-sized paper
%\usepackage[paper=a4paper,
%            %includefoot, % Uncomment to put page number above margin
%            marginparwidth=30.5mm,    % Length of section titles
%            marginparsep=1.5mm,       % Space between titles and text
%            margin=25mm,              % 25mm margins
%            includemp]{geometry}

%% More layout: Get rid of indenting throughout entire document
\setlength{\parindent}{0in}

% Provides special list environments and macros to create new ones
\usepackage[shortlabels]{enumitem}

% Simpler bibsections for CV sections
% (thanks to natbib for inspiration)
%
% * For lists of references with hanging indents and no numbers:
%
%   \begin{bibsection}
%       \item ...
%   \end{bibsection}
%
% * For numbered lists of references (with hanging indents):
%
%   \begin{bibenum}
%       \item ...
%   \end{bibenum}
%
%   Note that bibenum numbers continuously throughout. To reset the
%   counter, use
%
%   \restartlist{bibenum}
%
%   at the place where you want the numbering to reset.

\makeatletter
\newlength{\bibhang}
\setlength{\bibhang}{1em}
\newlength{\bibsep}
 {\@listi \global\bibsep\itemsep \global\advance\bibsep by\parsep}
\newlist{bibsection}{itemize}{3}
\setlist[bibsection]{label=,leftmargin=\bibhang,%
        itemindent=-\bibhang,
        itemsep=\bibsep,parsep=\z@,partopsep=0pt,
        topsep=0pt}
\newlist{bibenum}{enumerate}{3}
\setlist[bibenum]{label=[\arabic*],resume,leftmargin={\bibhang+\widthof{[999]}},%
        itemindent=-\bibhang,
        itemsep=\bibsep,parsep=\z@,partopsep=0pt,
        topsep=0pt}
\let\oldendbibenum\endbibenum
\def\endbibenum{\oldendbibenum\vspace{-.6\baselineskip}}
\let\oldendbibsection\endbibsection
\def\endbibsection{\oldendbibsection\vspace{-.6\baselineskip}}
\makeatother

%%% Setup header and footer (with page number and possible last page)
%
% The first block sets up pages 2--end
% The second block sets up page 1 formatting
%
%%%
%
% NOTE: comment the +LP lines and uncomment the -LP lines to have page
%       numbers without the ``of ##'' last page reference)
%
% NOTE: uncomment the \pagestyle{empty} line to get rid of all page
%       numbers on pages 2--end. To get rid of page numbers on page 1,
%       comment out the \thispagestyle{plain} line on the first page
%       below.
%       (also make sure includefoot is commented out above)
%
\usepackage{fancyhdr,lastpage}
\pagestyle{fancy}
%\pagestyle{empty}      % Uncomment this to get rid of page numbers
\fancyhf{}\renewcommand{\headrulewidth}{0pt}
\fancyfootoffset{\marginparsep+\marginparwidth}
\newlength{\footpageshift}
\setlength{\footpageshift}
          {0.5\textwidth+0.5\marginparsep+0.5\marginparwidth-2in}

%%%% PAGES 2--9 NUMBERING:
%% These two lines put page number in upper-right corner of pages 2--end
\rhead{Jung, M. - ~\arabic{page} of \protect\pageref*{LastPage}}   % +LP

%% These lines put page number in bottom (center) of pages 2--end
%\lfoot{\hspace{\footpageshift}%
%       \parbox{4in}{\, \hfill %
%                    \arabic{page} of \protect\pageref*{LastPage} % +LP
%%                    \arabic{page}                               % -LP
%                    \hfill \,}}
%%%% END PAGE 2--9 NUMBERING

%%%% PAGE 1 NUMBERING:
\makeatletter
\let\oldps@plain\ps@plain
\renewcommand{\ps@plain}{\oldps@plain%
\renewcommand{\@evenfoot}{\hspace*{-\footpageshift}\hfil %
    p.~\arabic{page} of \protect\pageref*{LastPage} % +LP
%    p.~\arabic{page}                               % -LP
    \hfil}%
\renewcommand{\@oddfoot}{\@evenfoot}}
\makeatother
%%%% END PAGE 1 NUMBERING

% Finally, give us PDF bookmarks and colored links
%
% NOTE: Some OCR software might be negatively affected by hyperlinks. So
%       most employers recommend the draft option here. Alternatively,
%       making all links black (as opposed to darkblue) should hopefully
%       prevent problems with most OCR.
%
% (to enable hyperlinks and bookmarks, comment out ``draft'' line;
%  to disable hyperlinks and bookmarks, uncomment ``draft'' line)
\usepackage{color,hyperref}
\definecolor{darkblue}{rgb}{0.0,0.0,0.3}
\hypersetup{breaklinks,colorlinks,
            linkcolor=black,urlcolor=black,
            anchorcolor=black,citecolor=black,
            %linkcolor=darkblue,urlcolor=darkblue,
            %anchorcolor=darkblue,citecolor=darkblue,
            %draft
            }

%%%%%%%%%%%%%%%%%%%%%%%% End Document Setup %%%%%%%%%%%%%%%%%%%%%%%%%%%%


%%%%%%%%%%%%%%%%%%%%%%%%%%% Helper Commands %%%%%%%%%%%%%%%%%%%%%%%%%%%%

%%% HEADING AT TOP OF CURRICULUM VITAE

% The title (name) with a horizontal rule under it
% (optional argument typesets an object right-justified across from name
%  as well)
%
% Usage: \makeheading{name}
%        OR
%        \makeheading[right_object]{name}
%
% Place at top of document. It should be the first thing.
% If ``right_object'' is provided in the square-braced optional
% argument, it will be right justified on the same line as ``name'' at
% the top of the CV. For example:
%
%       \makeheading[\emph{Curriculum vitae}]{Your Name}
%
% will put an emphasized ``Curriculum vitae'' at the top of the document
% as a title. Likewise, a picture could be included:
%
%   \makeheading[{\includegraphics[height=1.5in]{my_picture}}]{Your Name}
%
% the picture will be flush right across from the name. For this example
% to work, make sure the extra set of curly braces is included. Also
% makes ure that \usepackage{graphicx} is somewhere in the preamble.
\newcommand{\makeheading}[2][]%
        {\hspace*{-\marginparsep minus \marginparwidth}%
         \begin{minipage}[t]{\textwidth+\marginparwidth+\marginparsep}%
             {\large \bfseries #2 \hfill #1}\\[-0.15\baselineskip]%
                 \rule{\columnwidth}{1pt}%
         \end{minipage}}

%%% SECTION HEADINGS

% The section headings. Flush left in small caps down pseudo-margin.
%
% Usage: \section{section name}
\renewcommand{\section}[1]{\pagebreak[3]%
    \vspace{1.3\baselineskip}%
    \phantomsection\addcontentsline{toc}{section}{#1}%
    \noindent\llap{\scshape\smash{\parbox[t]{\marginparwidth}{\hyphenpenalty=10000\raggedright #1}}}%
    \vspace{-\baselineskip}\par}

%%% LISTS

% This macro alters a list by removing some of the space that follows the list
% (is used by lists below)
\newcommand*\fixendlist[1]{%
    \expandafter\let\csname preFixEndListend#1\expandafter\endcsname\csname end#1\endcsname
    \expandafter\def\csname end#1\endcsname{\csname preFixEndListend#1\endcsname\vspace{-0.6\baselineskip}}}

% These macros help ensure that items in outer-type lists do not get
% separated from the next line by a page break
% (they are used by lists below)
\let\originalItem\item
\newcommand*\fixouterlist[1]{%
    \expandafter\let\csname preFixOuterList#1\expandafter\endcsname\csname #1\endcsname
    \expandafter\def\csname #1\endcsname{\csname preFixOuterList#1\endcsname\let\oldItem\item\def\item{\pagebreak[2]\oldItem}}
    \expandafter\let\csname preFixOuterListend#1\expandafter\endcsname\csname end#1\endcsname
    \expandafter\def\csname end#1\endcsname{\let\item\oldItem\csname preFixOuterListend#1\endcsname}}
\newcommand*\fixinnerlist[1]{%
    \expandafter\let\csname preFixInnerList#1\expandafter\endcsname\csname #1\endcsname
    \expandafter\def\csname #1\endcsname{\let\oldItem\item\let\item\originalItem\csname preFixInnerList#1\endcsname}
    \expandafter\let\csname preFixInnerListend#1\expandafter\endcsname\csname end#1\endcsname
    \expandafter\def\csname end#1\endcsname{\csname preFixInnerListend#1\endcsname\let\item\oldItem}}

% An itemize-style list with lots of space between items
%
% Usage:
%   \begin{outerlist}
%       \item ...    % (or \item[] for no bullet)
%   \end{outerlist}
\newlist{outerlist}{itemize}{3}
    \setlist[outerlist]{label=\enskip\textbullet,leftmargin=*}
    \fixendlist{outerlist}
    \fixouterlist{outerlist}

% An environment IDENTICAL to outerlist that has better pre-list spacing
% when used as the first thing in a \section
%
% Usage:
%   \begin{lonelist}
%       \item ...    % (or \item[] for no bullet)
%   \end{lonelist}
\newlist{lonelist}{itemize}{3}
    \setlist[lonelist]{label=\enskip\textbullet,leftmargin=*,partopsep=0pt,topsep=0pt}
    \fixendlist{lonelist}
    \fixouterlist{lonelist}

% An itemize-style list with little space between items
%
% Usage:
%   \begin{innerlist}
%       \item ...    % (or \item[] for no bullet)
%   \end{innerlist}
\newlist{innerlist}{itemize}{3}
    \setlist[innerlist]{label=\enskip\textbullet,leftmargin=*,parsep=0pt,itemsep=0pt,topsep=0pt,partopsep=0pt}
    \fixinnerlist{innerlist}

% An environment IDENTICAL to innerlist that has better pre-list spacing
% when used as the first thing in a \section
%
% Usage:
%   \begin{loneinnerlist}
%       \item ...    % (or \item[] for no bullet)
%   \end{loneinnerlist}
\newlist{loneinnerlist}{itemize}{3}
    \setlist[loneinnerlist]{label=\enskip\textbullet,leftmargin=*,parsep=0pt,itemsep=0pt,topsep=0pt,partopsep=0pt}
    \fixendlist{loneinnerlist}
    \fixinnerlist{loneinnerlist}

%%% EXTRA SPACE

% To add some paragraph space between lines.
% This also tells LaTeX to preferably break a page on one of these gaps
% if there is a needed pagebreak nearby.
\newcommand{\blankline}{\quad\pagebreak[3]}
\newcommand{\halfblankline}{\quad\vspace{-0.5\baselineskip}\pagebreak[3]}

%%% FORMATTING MACROS

% Provides a linked \doi{#1} that links doi:#1 to http://dx.doi.org/#1
\usepackage{doi}
% To change the text before the DOI, adjust this command
%\renewcommand\doitext{doi:}

% Provides a linked \url{#1} that doesn't require escape characters
\usepackage{url}

% You can adjust the style \url{} uses here:
% (options are: same, rm, sf, tt; defaults to tt)
\urlstyle{same}

% For \email{ADDRESS}, links ADDRESS to the url mailto:ADDRESS
% (uncomment to typeset the e\-/mail address in typewriter font;
%  otherwise, will be typeset in the \urlstyle above)
%\DeclareUrlCommand\emaillink{\urlstyle{tt}}
\providecommand*\emaillink[1]{\nolinkurl{#1}}
\providecommand*\email[1]{\href{mailto:#1}{\emaillink{#1}}}

\providecommand\BibTeX{{B\kern-.05em{\sc i\kern-.025em b}\kern-.08em \TeX}}
\providecommand\Matlab{\textsc{Matlab}}

% Custom hyphenation rules for words that LaTeX has trouble with
\hyphenation{bio-mim-ic-ry bio-in-spi-ra-tion re-us-a-ble pro-vid-er Media-Wiki}

%%%%%%%%%%%%%%%%%%%%%%%% End Helper Commands %%%%%%%%%%%%%%%%%%%%%%%%%%%
%%%%%%%%%%%%%%%%%%%%%%%%% Begin CV Document %%%%%%%%%%%%%%%%%%%%%%%%%%%%

\begin{document}
\thispagestyle{plain}
\makeheading[\emph{Short CV}]{Martin Jung}

\section{Contact Information}

% NOTE: Mind where the & separators and \\ breaks are in the following
%       table. Table is one row made up of three parboxes. The left
%       parbox has address info, the middle parbox has a vertical bar,
%       and the right parbox has phone and electronic contact
%       information.
%
% MACROS: \rcollength is the width of the right column of the table
%             (adjust it to your liking; default is 1.85in).
%         \spacewidth is width of area between left and right boxes.
%
\newlength{\rcollength}\setlength{\rcollength}{2.75in}%
\newlength{\spacewidth}\setlength{\spacewidth}{20pt}
%
\begin{tabular}[t]{@{}p{\textwidth-\rcollength-\spacewidth}@{}p{\spacewidth}@{}p{\rcollength}}%

% Address box
\parbox{\textwidth-\rcollength-\spacewidth}{%
Martin Jung \\
School of Life Sciences\\
\href{http://www.sussex.ac.uk}{University of Sussex}\\
%John Maynard Smith (JMS) building \\
BN1 9RH Brighton \\
United Kingom 
}

&
% Uncomment to add a vertical bar in middle of contact information
{\vrule width 0.1pt}
\parbox[m][5\baselineskip]{\spacewidth}{} &

% Non-snail-mail contact information
\parbox{\rcollength}{%
\emph{\faMobilePhone} {\small +44 07761868701} \\
\emph{\faEnvelope} {\small \email{m.jung@sussex.ac.uk}}\\
\emph{\faGlobe} {\small  \href{http://martin-jung.github.io}{martin-jung.github.io}}\\
\emph{\faLinkedinSquare} { \small \href{https://linkedin.com/in/jungmartin}{linkedin.com/in/jungmartin} } \\
\emph{\faTwitter} {\small \href{https://twitter.com/Martin\_Ecology}{twitter.com/Martin\_Ecology}}
}

\end{tabular}
% -------------------------------------------- %
% Career so far %

\section{Career history}

\begin{lonelist}

\item{\cventry{January 2017--July 2018}{Associate doctoral tutor}{University of Sussex}{Brighton, UK}{}{\vspace{3pt} Assisted in the teaching and marking of biology and geography courses for BSc and MSc students. Led one day intensive workshops in GIS \& data analysis for graduate students.}}

\vspace{4pt}

\item{\cventry{February 2015--September 2015}{Research assistant}{Lund university - Centre for Sustainability studies}{Lund, Sweden}{}{\vspace{3pt} Worked for the \href{http://www.kimnicholas.com/uploads/2/5/7/6/25766487/brochure.pdf}{AGROMES} project (funded by the Swedish research council) mapping the environmental, economic,and social tradeoffs of European farming systems across scales. Involved the acquisition, manipulation, analysis and visualisation of economic and environmental data. Collaboration with the University of British Columbia and Sussex University }}

\vspace{4pt}

\item{\cventry{October 2014--August 2015}{External database contractor}{Imperial College London}{London, UK}{}{\vspace{3pt} 
Developed, implemented and administrated a spatially explicit relational database (PostgreSQL \& PostGIS) for the \href{https://biofrag.wordpress.com/}{BIOFRAG} project. Writing parsing scripts and SQL queries to maintain the database.}}

\vspace{4pt}

\item{\cventry{May 2013--April 2015}{Laboratory assistant}{University of Copenhagen}{Copenhagen, Denmark}{}{\vspace{3pt} 
Laboratory and field work as well as statistical analyses for biological pest control (Fruitgrowth, Softpest, INBIOsoil) projects. }}

\vspace{4pt}

\item{\cventry{October 2012--September 2012}{Research assistant}{gaiac - research institute}{Aachen, Germany}{}{\vspace{3pt} 
GIS and statistical analyses around the Virtual Forest project and urban ecology. Plant vegetation surveys on windfall calamity areas}}

\vspace{4pt}

\item{\cventry{July 2006--May 2013}{Part-time database maintainer}{Solarenergie-F{\"o}rderverein Deutschland e.V.,}{Aachen, Germany}{}{\vspace{3pt} 
Data maintenance and user support of the largest database for private solar panel owners.}}

\end{lonelist}

% ------------------------------------------%
% Education

\section{Education}

\begin{lonelist}
    \item[] \textbf{PhD, Environmental Science}, \href{http://www.sussex.ac.uk/lifesci/scharlemannlab}
             {University of Sussex, UK} \\ 
             \hspace*{\fill}  \emph{September 2015 - March 2019}
    

    \item[] \textbf{MSc, Biology},
    \href{http://www.science.ku.dk/english/}
             {University of Copenhagen, Denmark} \\
             \hspace*{\fill}  \emph{February 2013 - April 2015}
             

    \item[] \textbf{BSc, Biology},
    \href{http://www.uni-marburg.de/fb17/}
    {University of Marburg, Germany} \\
             \hspace*{\fill}  \emph{October 2009 - October 2012}
             
             
\end{lonelist}

\vspace{4pt}


% \section{Professional Experience}
%
% \href{http://www.lunduniversity.lu.se/home}{\textbf{Lund University - Centre for Sustainability Studies}},\\
% Lund, Sweden
% \begin{outerlist}
%
%     \item[] \textit{Research assistant}%
%             \hfill \textbf{February 2015 to September 2015}
%             \begin{innerlist}
%                 \item Assisting within the AGROMES project by mapping the environmental, economic, and social tradeoffs of European farming systems across scales. Involves the acquisition, manipulation, analysis and visualization of multiple data sources. Conducted in collaboration with the University of British Columbia and Sussex University. 
%             \end{innerlist}
%
% \end{outerlist}
%
% \halfblankline
%
% \href{http://forestecology.net}{\textbf{Imperial College London - Forest Ecology and Conservation Research Group}},\\
% London, UK
% \begin{outerlist}
%
%     \item[] \textit{External contractor}%
%             \hfill \textbf{October 2014 to present}
%
% \end{outerlist}
%
% \halfblankline
%
% \href{http://plen.ku.dk/english/}{\textbf{University of Copenhagen - Department of Plant and Environmental Science}},\\
% Copenhagen, Denmark
% \begin{outerlist}
%
%     \item[] \textit{Laboratory assistant}%
%             \hfill \textbf{May 2013 to April 2015}
%
% \end{outerlist}
%
% \halfblankline
%
%
% \href{http://www.gaiac.rwth-aachen.de/index.php/en.html}{\textbf{gaiac - Research institute for ecosystem analysis and assessment}},\\
% Aachen, Germany
% \begin{outerlist}
%
%     \item[] \textit{Research assistant}%
%             \hfill \textbf{October 2012 to January 2013}
%
% \end{outerlist}
%
% \halfblankline
%
% \href{http://www.ufz.de/index.php?en=1625}{\textbf{UFZ - Department for Conservation Biology}},\\
% Leipzig, Germany
% \begin{outerlist}
%
%     \item[] \textit{Internship}%
%             \hfill \textbf{September 2011 to October 2011}
%
% \end{outerlist}
%
% \halfblankline
%
% \href{https://www.uni-marburg.de/fb17/fachgebiete/oekologie/conserv_ecol/staff/joerg-albrecht?set_language=en}{\textbf{University of Marburg}},\\
% Bia\l owie\.{z}a, Poland
% \begin{outerlist}
%
%     \item[] \textit{Volunteer and Research assistant}%
%             \hfill \\ \textbf{July 2011 to August 2011 and April 2012 - July 2012}
% \end{outerlist}
%
% \halfblankline
%
% \href{http://www.save-wildlife.com/en}{\textbf{SAVE Wildlife Conservation Fund}},\\
% W\"{u}lfrath, Germany
% \begin{outerlist}
%
%     \item[] \textit{Volunteer Project Leader}%
%             \hfill \textbf{Mai 2011 to September 2012}
%
% \end{outerlist}
%
% \halfblankline
%
% \href{www.nlwkn.niedersachsen.de}{\textbf{Lower Saxon State Department for Waterway, Coastal and Nature Conservation}},\\
% Langeoog, Germany
% \begin{outerlist}
%
%     \item[] \textit{Community Service and assistant park ranger}%
%             \hfill \textbf{July 2008 to June 2009}
%
% \end{outerlist}
%
% \halfblankline
%
% \href{http://www.sfv.de/}{\textbf{Solarenergie-F\"{o}rderverein Deutschland e.V.}},\\
% Aachen, Germany
% \begin{outerlist}
%
%     \item[] \textit{Database maintainer}%
%             \hfill \textbf{July 2006 to May 2013}
%
% \end{outerlist}
%
% \section{Manuscripts in preperation}
%
%  \begin{bibenum}
%
%     \item \textbf{Jung,M.}, Rowhani,P. and Scharlemann,J. (in prep) "The impact of major disturbances on biodiversity increases on more anthropogenic used land", \emph{Proceedings of the National Academy of Sciences}
%
%     \item \textbf{Jung,M.}, Rowhani,P., Newbold, T., Purvis, A. and Scharlemann,J. (in prep) "Past conditions in land cover and land use influence local species assemblages more than current conditions", \emph{Ecography}
%
%   \end{bibenum}
%

\section{Scientific publications}

\begin{itemize}
    \item[] https://scholar.google.co.uk/citations?user=qa-Dz6QAAAAJ&hl=en&oi=sra
\end{itemize}


% -------------------------------------------------------- %
%                       Skills                             %
\section{Professional skills}
\vspace{6pt}
\begin{itemize}

    \item \textbf{Languages:} Fluent in German \& English, basic literacy in Danish.
    \item \textbf{General:} Very good project and time management skills. Experienced in giving presentations, working well solo or in a team.
    \item \textbf{Statistics:} Univariate and multivariate statistics. Knowledge about bayesian and modern machine learning techniques. Time series analyses.
    \item \textbf{Coding:} Proficient in: R, Python, Bash, Javascript, SQL, HTML, CSS, TeX, Learning Julia.
    \item \textbf{Databases:} MySQL, PostgreSQL, PostGIS, SQLite, Microsoft Access.
    \item \textbf{General computer skills:} Broad Linux \& Microsoft Windows knowledge. Most existing GIS software. Comfortable with big data. Cloud computing knowledge (Google Cloud, Google Earth Engine, AWS). Excellent command of Microsoft office.
\end{itemize}


\end{document}
